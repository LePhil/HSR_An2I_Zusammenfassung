%%%%%%%%%%%%%%%%%%%%%%%%%%%%%%%%%%%%%%%%%%%%%%%%%%%%%%%%%%%%%%%%%%%%%%%%
\section{Integralrechnung}
\subsection{Grundlagen}
\subsubsection{Bestimmtes Integral}
Sei $f$ eine auf dem Intervall $[a;b]$ definierte Funktion.
\[ \int^b_a f = \int^b _a f(x) dx =
  \lim_{n \to \infty} \frac{b-a}{n} \sum_{k=1}^{n}f(x_k) \cdot \Delta x
\]
mit
\[ f(x_k) = a + k\left(\frac{b-a}{n}\right) \text{ und }
  x_k = a + i \cdot \Delta x\]
existiert, dann heisst die Funktion auf dem Intervall $[a;b]$
integrierbar.

\subsubsection{Graphische Interpretation von Integralen}
Beim Integral $\int_{a}^{b}f$ gibt es zwei Fälle:
\begin{itemize}
  \item Wenn $a < b$: Positive Ordinate zählt positiv;
    Negative Ordinate zählt negativ
  \item Wenn $a > b$: Positive Ordinate zählt negativ;
    Negative Ordinate zählt positiv
\end{itemize}

\subsubsection{Grundregeln für Integrale}
Faktorregel:
\[ \int_{a}^{b}c \cdot g = c \cdot \int_{a}^{b}f \]
Vertauschen der Integralgrenzen ändert das Vorzeichen des Integrals:
\[ \int_{a}^{b}f = - \int_{b}^{a}f \]
Aneinanderstossende Integrale können zusammengefasst werden:
\[ \int_{a}^{b}f + \int_{b}^{c}f = \int_{a}^{c}f \]
Linearität:
\[ \int_{a}^{b}f+g = \int_{a}^{b}f + \int_{a}^{b}g \]
Gleiche Integrationsgrenzen
\[ \int_{a}^{a} f = 0 \]

\subsubsection{Nummerische Berechnung von Integralen}
Rechteckregel:
\[ \int^b_a f =  \Delta x \sum_{k=1}^{n}f(x_k) \]
mit
\[ \Delta x = \frac{b-a}{n} \text{ und } x_k = a + i \cdot \Delta x \]

\subsection{Berechnung von Integralen mit Stammfunktionen}
\subsubsection{Integralfunktion}
\[ \phi_a(x) = \int_a^x f \]
\begin{itemize}
  \item Die Integralfunktion hängt vom Parameter $a$ ab.
  \item Ändert man den Parameter $a$, so ändert sich die
  Integralfunktion nur um eine Konstante ($\phi_b(x) = \phi_a(x) + C$),
  was eine Verschiebung auf der Y-Achse bewirkt.
\end{itemize}

Aus der Ableitung der Integralfunktion erhalten wir die ursprüngliche Funktion:
\[ \frac{dx}{x} \phi_a(x) = \frac{dx}{x} \int_a^x f = f(x) \]

\subsubsection{Stammfunktion}
Wir nennen eine Funktion $F$ Stammfunktion von $f$, wenn die Ableitung
der Stammfunktion $f$ ergibt:
\[F \text{ mit } F' = f\]
\subsubsection{Hauptsatz}
Jede Integralfunktion ist eine Stammfunktion. Diese können wir zum
Berechnen von Integralen benutzen.
\[\phi_a(x) = \int_a^x f  = F(x) - F(a) = F(x)|^b_a\]
% LaTeX TODO: Pipe grösser machen

\subsubsection{Unbestimmtes Integral}
Das unbestimmte Integral $\int f$ ist die Menge aller Stammfunktionen
von $f$.
\[ \int f(x) dx = F(x) + C \text{ bzw. } \int f = F + C\]

\subsubsection{Rechenregeln}
Verkettung linearer Funktionen
\[ \int f(ax+b)dx = \frac{1}{a}F(ax+b) + c \]

Spezialfall der Produkteregel
\[ \int f(x) \cdot f'(x) dx = \frac{1}{2}f^2(x) + c \]

Quotientenregel
\[ \int \frac{f'(x)}{f(x)}dx = ln(|f(x)|) \]

Substitutionsregel
\[ \int f(g(x)) \cdot f'(x) dx = G(f(x)) + c\]


% %%%%%%%%%%%%%%%%%%%%%%%%%%%%%%%%%%%%%%%%%%%%%%%%%%%%%%%%%%%%%%%%%%%%%%%%
% \section{Fouriertransformation}

% %%%%%%%%%%%%%%%%%%%%%%%%%%%%%%%%%%%%%%%%%%%%%%%%%%%%%%%%%%%%%%%%%%%%%%%%
% \section{Differenzialgleichungen}

%%%%%%%%%%%%%%%%%%%%%%%%%%%%%%%%%%%%%%%%%%%%%%%%%%%%%%%%%%%%%%%%%%%%%%%%
\section{Anhang}
\subsection{Summenformeln}
\[ \sum_{i=1}^{n}i = \frac{n(n+1)}{2} \]
\[ \sum_{i=1}^{n}i^2 = \frac{n(n+1)(2n+1)}{6} \]
\[ \sum_{i=1}^{n}i^3 = \left(\frac{n(n+1}{2}\right)^2 \]
