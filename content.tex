%%%%%%%%%%%%%%%%%%%%%%%%%%%%%%%%%%%%%%%%%%%%%%%%%%%%%%%%%%%%%%%%%%%%%%%%
\section{Integralrechnung}
\subsection{Bestimmtes Integral}
Sei $f$ eine auf dem Intervall $[a;b]$ definierte Funktion.
\[ \int^b_a f = \int^b _a f(x) dx =
  \lim_{n \to \infty} \frac{b-a}{n} \sum_{k=1}^{n}f(x_k) \cdot \Delta x
\]
mit
\[ f(x_k) = a + k\left(\frac{b-a}{n}\right) \text{ und }
  x_k = a + i \cdot \Delta x\]
existiert, dann heisst die Funktion auf dem Intervall $[a;b]$
integrierbar.

\subsection{Graphische Interpretation von Integralen}
Beim Integral $\int_{a}^{b}f$ gibt es zwei Fälle:
\begin{itemize}
  \item Wenn $a < b$: Positive Ordinate zählt positiv;
    Negative Ordinate zählt negativ
  \item Wenn $a > b$: Positive Ordinate zählt negativ;
    Negative Ordinate zählt positiv
\end{itemize}

\subsection{Rechnen mit Integralen}
Vertauschen der Integralgrenzen ändert das Vorzeichen des Integrals:
\[ \int_{a}^{b}f = - \int_{b}^{a}f \]
Aneinanderstossende Integrale können zusammengefasst werden:
\[ \int_{a}^{b}f + \int_{b}^{c}f = \int_{a}^{c}f \]
Es gelten die Linearitätsgesetze
\[ \int_{a}^{b}f+g = \int_{a}^{b}f + \int_{a}^{b}g \]
und
\[ \int_{a}^{b}c \cdot g = c \cdot \int_{a}^{b}f \]

\subsection{Nummerische Berechnung von Integralen}
\subsubsection{Rechteckregel}
\[ \int^b_a f =  \Delta x \sum_{k=1}^{n}f(x_k) \]
mit
\[ \Delta x = \frac{b-a}{n} \text{ und } x_k = a + i \cdot \Delta x \]

\subsubsection{Trapezregel}
\subsubsection{Simpson-Regel (Fassregel)}


%%%%%%%%%%%%%%%%%%%%%%%%%%%%%%%%%%%%%%%%%%%%%%%%%%%%%%%%%%%%%%%%%%%%%%%%
\section{Anhang}
\subsection{Summenformeln}
\[ \sum_{i=1}^{n}i = \frac{n(n+1)}{2} \]
\[ \sum_{i=1}^{n}i^2 = \frac{n(n+1)(2n+1)}{6} \]
\[ \sum_{i=1}^{n}i^3 = \left(\frac{n(n+1}{2}\right)^2 \]
