%%%%%%%%%%%%%%%%%%%%%%%%%%%%%%%%%%%%%%%%%%%%%%%%%%%%%%%%%%%%%%%%%%%%%%%%
\section{Integralrechnung}
\subsection{Grundlagen}
\subsubsection{Bestimmtes Integral}
Sei $f$ eine auf dem Intervall $[a;b]$ definierte Funktion.
\[ \int^b_a f = \int^b _a f(x) dx =
  \lim_{n \to \infty} \frac{b-a}{n} \sum_{k=1}^{n}f(x_k) \cdot \Delta x
\]
mit
\[ f(x_k) = a + k\left(\frac{b-a}{n}\right) \text{ und }
  x_k = a + i \cdot \Delta x\]
existiert, dann heisst die Funktion auf dem Intervall $[a;b]$
integrierbar.

\subsubsection{Graphische Interpretation von Integralen}
Beim Integral $\int_{a}^{b}f$ gibt es zwei Fälle:
\begin{itemize}
  \item Wenn $a < b$: Positive Ordinate zählt positiv;
    Negative Ordinate zählt negativ
  \item Wenn $a > b$: Positive Ordinate zählt negativ;
    Negative Ordinate zählt positiv
\end{itemize}

\subsubsection{Grundregeln für Integrale}
\begin{itemize}
  \item Faktorregel: $\int_{a}^{b}c \cdot g = c \cdot \int_{a}^{b}f$
  \item Vertauschen der Integralgrenzen ändert das Vorzeichen des Integrals:
    $\int_{a}^{b}f = - \int_{b}^{a}f$
  \item Aneinanderstossende Integrale können zusammengefasst werden:
    $\int_{a}^{b}f + \int_{b}^{c}f = \int_{a}^{c}f$
  \item Linearität: $\int_{a}^{b}f+g = \int_{a}^{b}f + \int_{a}^{b}g$
  \item Gleiche Integrationsgrenzen $\int_{a}^{a} f = 0$
\end{itemize}

\subsubsection{Nummerische Berechnung von Integralen}
Rechteckregel:
\[ \int^b_a f =  \Delta x \sum_{k=1}^{n}f(x_k) \]
mit
\[ \Delta x = \frac{b-a}{n} \text{ und } x_k = a + i \cdot \Delta x \]

\subsection{Berechnung von Integralen mit Stammfunktionen}
\subsubsection{Integralfunktion}
\[ \phi_a(x) = \int_a^x f \]
\begin{itemize}
  \item Die Integralfunktion hängt vom Parameter $a$ ab.
  \item Ändert man den Parameter $a$, so ändert sich die
  Integralfunktion nur um eine Konstante ($\phi_b(x) = \phi_a(x) + c$),
  was eine Verschiebung auf der Y-Achse bewirkt.
\end{itemize}

Aus der Ableitung der Integralfunktion erhalten wir die ursprüngliche Funktion:
\[ \frac{dx}{x} \phi_a(x) = \frac{dx}{x} \int_a^x f = f(x) \]

\subsubsection{Stammfunktion}
Wir nennen eine Funktion $F$ Stammfunktion von $f$, wenn die Ableitung
der Stammfunktion $f$ ergibt:
\[F \text{ mit } F' = f\]
\subsubsection{Hauptsatz}
Jede Integralfunktion ist eine Stammfunktion. Diese können wir zum
Berechnen von Integralen benutzen.
\[\phi_a(b) = \int_a^b f(x)dx = F(x)|^b_a = F(b) - F(a) \]
% LaTeX TODO: Pipe grösser machen

\subsubsection{Unbestimmtes Integral}
Das unbestimmte Integral $\int f$ ist die Menge aller Stammfunktionen
von $f$.
\[ \int f(x) dx = F(x) + c \text{ bzw. } \int f = F + c\]

\subsubsection{Rechenregeln}
\begin{itemize}
  \item Verkettung linearer Funktionen: $\int f(ax+b)dx = \frac{1}{a}F(ax+b) + c$
  \item Produkteregel: $\int_a^b f'(x) \cdot g(x) dx =
    f(x) \cdot g(x)|^b_a - \int_a^b f(x) \cdot g'(x) dx$
  \item Spezialfall der Produkteregel: $\int f(x) \cdot f'(x) dx = \frac{1}{2}f^2(x) + c$
  \item Quotientenregel: $\int \frac{f'(x)}{f(x)}dx = ln(|f(x)|) + c$
  \item Substitutionsregel: $ \int f(g(x)) \cdot f'(x) dx = G(f(x)) + c$
\end{itemize}


% %%%%%%%%%%%%%%%%%%%%%%%%%%%%%%%%%%%%%%%%%%%%%%%%%%%%%%%%%%%%%%%%%%%%%%%%
\section{Fouriertransformation}
\subsection{Fourierreihen}
Eine Fourierreihe der Funktion $f$ besteht aus einer Linearkombination
von Sinus- und Kosinus-Funktionen, welche alle dieselbe Persiode $T$
haben. Je höher die Ordnung $n$, desgo genauer wird die Funktion $f$ aproximiert.

\subsubsection{Sinus-Kosinus-Form}
\[
  f(t) = a_0 + \sum_{k=1}^{n}
  (a_k \cdot cos(k \omega_1 t) + b_k \cdot sin(k \omega_1 t))
\]
\begin{itemize}
  \item Grundkreisfrequenz (Gemeinsame Periode) $\omega_1 = \frac{2\pi}{T}$
  \item Konstante $a_0 = A_0 = \frac{1}{T}\int_0^T s(t) dt$
  \item Koeffizient $a_k = A_k \cdot cos(\phi_k) =
    \frac{2}{T} \int_0^T s(t) \cdot cos(k \omega_1 t) dt$
  \item Koeffizient $b_k = A_k \cdot sin(\phi_k) =
    \frac{2}{T} \int_0^T s(t) \cdot sin(k \omega_1 t) dt$
\end{itemize}

\subsubsection{Amplituden-Phasen-Form}
\[
  f(t) = A_0 + \sum_{k=1}^{n}
  (A_k \cdot cos(k \omega_1 t  - \phi_k))
\]
\begin{itemize}
  \item Grundkreisfrequenz (Gemeinsame Periode) $\omega_1 = \frac{2\pi}{T}$
  \item Konstante $A_0 = a_0$
  \item Koeffizient $A_k = \sqrt{a_k^2 + b_k^2}$
  \item $\phi_k =  \begin{cases}
      arctan\left(\frac{b_k}{a_k}\right) & \text{ für } a_k > 0 \\
      arctan\left(\frac{b_k}{a_k}\right) + \pi & \text{ für } a_k < 0 \\
      \frac{\pi}{2} & \text{ für } a_k = 0 \wedge b_k > 0\\
      -\frac{\pi}{2} & \text{ für } a_k = 0 \wedge b_k < 0\\
    \end{cases}$
\end{itemize}

\subsection{Eigenschaften}
\subsubsection{Gerade und ungerade Funktionen}
\paragraph{Gerade Funktionen}
\begin{itemize}
  \item Für die geraden Funktionen ist das Integral
    $\int_0^{\frac{T}{2}}$ am optimalsten.
  \item Koeffizient $b_k = 0$
\end{itemize}
\paragraph{Ungerade Funktionen}
\begin{itemize}
  \item Für die ungeraden Funktionen ist das Integral
    $\int_{-\frac{T}{2}}^{\frac{T}{2}}$ am optimalsten.
  \item Koeffizient $a_k = 0$
\end{itemize}

\subsubsection{Transformation von Fourierreihen}
\paragraph{Spiegeln an X-Achse}
\begin{itemize}
  \item Sinus-Kosinus-Form: $r(t) = -s(t) = -a_0 + \sum_{k=1}^{n} ( -a_k \cdot
    cos(k \omega_1 t) + (-b_k) \cdot sin(k \omega_1 t))$
  \item Amplituden-Phasen-Form: $r(t) = -s(t) = -A_0 + \sum_{k=1}^{n} (A_k \cdot cos(k
  \omega_1 t  - (\phi_k + \pi)))$
\end{itemize}
\paragraph{Spiegeln an Y-Achse}
\begin{itemize}
  \item Sinus-Kosinus-Form: $r(t) = s(-t) = a_0 + \sum_{k=1}^{n} ( a_k \cdot
    cos(k \omega_1 t) + (-b_k) \cdot sin(k \omega_1 t))$
  \item Amplituden-Phasen-Form: $r(t) = s(-t) =  A_0 + \sum_{k=1}^{n}
    (A_k \cdot cos(k \omega_1 t  - (-\phi_k)))$
\end{itemize}
\paragraph{Skalierung auf der X-Achse (Zeitskalierung)}
\begin{itemize}
  \item Die Zeitskalierung beeinflusst nur die Grundkreisfrequenz,
    welche man mit der Formel neu berechnen kann.
\end{itemize}


% %%%%%%%%%%%%%%%%%%%%%%%%%%%%%%%%%%%%%%%%%%%%%%%%%%%%%%%%%%%%%%%%%%%%%%%%
% \section{Differenzialgleichungen}

%%%%%%%%%%%%%%%%%%%%%%%%%%%%%%%%%%%%%%%%%%%%%%%%%%%%%%%%%%%%%%%%%%%%%%%%
\section{Anhang}
\subsection{Summenformeln}
\[ \begin{aligned}
  \sum_{i=1}^{n}i     & = \frac{n(n+1)}{2} \\
  \sum_{i=1}^{n}i^2   & = \frac{n(n+1)(2n+1)}{6} \\
  \sum_{i=1}^{n}i^3   & = \left(\frac{n(n+1}{2}\right)^2 \\
\end{aligned} \]
